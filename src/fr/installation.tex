% !TEX encoding = UTF8
% !TEX root     = manuel.tex

\chapter{Installation}
\index{installation}
\label{chap.installation}

\Tw{} n'est qu'un éditeur de texte; pour pouvoir créer des documents avec \AllTeX{} et les composer en PDF, nous avons besoin de ce qu'on appelle une distribution \TeX\index{TeX distribution@{\TeX} distribution}\index{TeX@\TeX!distribution|see {{\TeX} distribution}}. C'est un ensemble de programmes et autres fichiers complémentaires qui seront appelés automatiquement par \Tw{} durant son travail. Il faut donc installer une distribution; nous le ferrons \emph{avant} de lancer \Tw{} la première fois pour que celui-ci trouve automatiquement ce dont il a besoin.

\textbf{TeX~Live\index{TeX distribution@{\TeX} distribution!TeX~Live}} (\url{https://tug.org/texlive/}), une combinaison de teTex, MacTeX et XEmTeX, est disponible pour les trois systèmes d'exploitation (Linux, macOS, Windows).

\begin{OSLinux}
Pour Linux\index{TeX distribution@{\TeX} distribution!Linux}: la plupart des distributions ont une distribution \TeX, elle peut cependant ne pas être installée de base et il faudra utiliser les outils de gestion Linux pour le faire. En outre, on peut télécharger et installer TeX~Live directement à partir de \url{https://tug.org/texlive/}.
\end{OSLinux}
\vspace{6pt} 

\begin{OSMac}
Pour le Mac\index{TeX distribution@{\TeX} distribution!Mac}: \textbf{MacTeX}\index{TeX distribution@{\TeX} distribution!Mac!MacTeX}, une distribution basée sur gwTeX et XeTeX est disponible; voir \url{https://tug.org/mactex/}.
\end{OSMac}
\vspace{6pt} 

\begin{OSWindows}
Pour Windows\index{TeX distribution@{\TeX} distribution!Windows}: une distribution très populaire est \textbf{MiKTeX\index{TeX distribution@{\TeX} distribution!Windows!MikTeX}} (\url{https://miktex.org/}). MiKTeX dispose d'un programme de mise-à-jour de la distribution qui a aussi été portée sur Linux.
\end{OSWindows}

Pour des détails sur une usage portable et le changement des localisations de la configuration locale, veuillez vous reportez à la section Usage Portable \& Modifier la configuration -- section \ref{sec.portable_configuration} (à la page \pageref{sec.portable_configuration}).

\section{Sous Windows}\index{installation!Windows}

La plupart des grandes distributions \TeX{} ont déjà un module \Tw. Parfois ces versions ont des améliorations spécifiques pour la distribution. C'est pourquoi il est préférable pour installer \Tw{} sur Windows d'utiliser le gestionnaire de modules de votre distribution. Dans ce cas, vous pouvez sauter les quelques paragraphes qui suivent. Mais, cependant, lisez la fin de cette section car elle fournit des indications sur la manière d'adapter \Tw{} à vos besoins.

Si vous voulez obtenir une version \og officielle\fg, téléchargez le système setup de \Tw{} à partir de  \url{https://tug.org/texworks/}  après installation de la distribution \TeX.

Installer simplement \Tw{} en exécutant le fichier setup. Durant l'installation, on vous demandera où vous désirez installer le programme, si vous désirez créer des raccourcis et si vous voulez toujours ouvrir les fichiers \path{.tex} avec \Tw. Il y a des valeurs par défaut correctes qui devraient fonctionner pour la plupart des utilisateurs.

Si vous voulez avoir un contrôle complet sur où et comment \Tw{} est localisé, vous pouvez aussi télécharger l'archive \path{.zip} à partir du site web et la désarchiver où voulez. Notez que dans ce cas les raccourcis et les associations de fichiers doivent être créés manuellement.

\urldef{\TwRegistryPath}\path{\HKEY_CURRENT_USER\Software\TUG\TeXworks}

Lorsque vous lancez {\Tw} pour la première fois, il crèe un dossier appelé \path{C:\Users\<votre nom>\AppData\Roaming\TUG\TeXworks}\index{folder!resource}. Ce dossier contiendra quelques sous-dossiers pour les fichiers d'auto-complétion\index{folder!auto-completion}, de configuration\index{folder!configuration}, des dictionnaires\index{folder!dictionaries}, des modèles\index{folder!templates}, et de la traduction/localisation de l'interface\index{folder!translations} ---nous verrons cela en plus de détails plus loin.\footnote{{\Tw} sauvegardera ses préférences dans le registre:
\TwRegistryPath. Si elles sont supprimés, elles seront recréés avec des valeurs par défaut à l'utilisation suivante.}.

NB. Au moment de la rédaction, si \path{<your name>} contient des caractères non-ASCII (par exemple des caractères accentués), quelque fonctions de {\Tw} pourrais ne pas fonctionner correctement. Par exemple, la vérification orthographique et la synchronisation aller/retour entre la source et le \path{.pdf} seront affectés.

\section{Sous Linux}\index{installation!Linux}

Plusieurs distributions Linux courantes ont déjà des modules \Tw. Ils sont adaptés à la plupart des utilisateurs et facilitent considérablement l'installation de \Tw.

Si votre distribution ne fournit pas de modules adéquats et récents, vous devrez construire vous-mêmes \Tw{} à partir des sources, ce qui est vraiment facile sous Linux. Après installation de la distribution \TeX, allez à \url{https://github.com/TeXworks/texworks/wiki/Building} et suivez les instructions adaptées à votre distribution Linux. Voyez aussi la section \ref{sec.compiling}.

Une fois le programme installé, lancez {\Tw}. Les dossiers \path{.local/share/TUG/TeXworks}\index{folder!resource} et \path{.config/TUG} seront créés dans votre dossier.

\section{Sous macOS}\index{installation!Mac}

Si vous désirez obtenir une version \og officielle\fg{}, récupérez \Tw{} en téléchargeant l'archive du siteg de \Tw{} \url{https://tug.org/texworks/} après installation de la distribution {\TeX}.

C'est un module autonome, \texttt{.app}, qui ne requière pas l'installation de fichiers Qt dans \path{/Library/Frameworks}, ou d'autres librairies dans \path{/usr/local/lib}. Copiez juste l'\path{.app} où vous voulez et lancez le.

Sur macOS, le dossier ressource de {\Tw} \index{folder!resource} sera créé dans votre dossier \path{Library} (\path{~/Library/Application Support/TUG/TeXworks}), à l'intérieur de votre dossier home. Les préférences sont enregistrées dans \path{~/Library/Preferences/org.tug.TeXworks.plist} 
que vous pouvez supprimer si jamais vous suspectez qu'il cause des problèmes

\section{Prêts!}

Enfin, on pourait devoir ajouter quelques fichiers pourraient aux fichiers \og personal\fg{} que {\Tw} crée. Comme la localisation exacte de ceux-ci dépend de votre plateforme, on y fera référence comme \path{<resources>}\index{folder!resource} ou le \textbf{dossier ressource de {\Tw}} dans le cours de ce manuel. Par défaut sous Windows, c'est \path{C:\Users\<your name>\AppData\Roaming\TUG\TeXworks}, sous Linux c'est \path{.local/share/TUG/TeXworks}, et sous macOS c'est \path{~/Library/Application Support/TUG/TeXworks}. Le moyen le plus facile pour localiser ce dosiier dans les versions récentes de {\Tw} est d'utiliser l'entrée de \menu{Aide}\submenu\menu{Paramètres et Ressources\dots}. Elle ouvrer une boîte de dialogue qui montre où {\Tw} sauvegarde ses paramètres et où il cherche ses ressources.

Après l'installation et la première utilisation, jetez un coup d'œil dans les sous-dossiers du dossier ressources de {\Tw} et supprimer tout fichier \path{qt_temp.xxxx} files; ce sont des fichiers temporaires abandonnés et ils pourraient interférer avec les fichiers normaux, qui sont installés, par après, dans le même dossier.
