% !TEX encoding   = UTF8
% !TEX root       = manuel.tex
% !TEX spellcheck = fr_FR

\chapter{Compiler {\Tw}}\index{compiler {\Tw}}
\label{sec.compiling}

Un guide complet sur comment compiler \Tw{} dépasse le but de ce manuel. Cependant, la plupart des utilisateurs devraient trouver des versions précompilées adaptées à leur système, soit inclues dans leur distribution \TeX, soit avec leur système. Si ce n'est pas le cas, différentes versions précompilées sont disponibles en téléchargement à partir de \url{http://www.tug.org/texworks/}.

Compiler \Tw{} vous-même n'est seulement nécessaire que si votre système n'est pas encore supporté, si vous voulez toujours avoir les dernières caractéristiques (et bogues) ou si vous  désirez aider à encore améliorer \Tw. À cet effet, il y a des documents donnant des instructions détaillées pour compiler \Tw{} sur différentes machines.

\begin{OSLinux}
\noindent\url{https://github.com/TeXworks/texworks/wiki/Building} \\
\end{OSLinux}

\begin{OSMac}
\noindent\raggedright\url{https://github.com/TeXworks/texworks/wiki/Building-on-Mac-OS-X-(Homebrew)} \\
\end{OSMac}

\begin{OSWindows}
\noindent\raggedright\url{https://github.com/TeXworks/texworks/wiki/Building-on-Windows-(MinGW)} \\
\end{OSWindows}
