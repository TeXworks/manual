% !TEX encoding   = UTF8
% !TEX root       = manuel.tex
% !TEX spellcheck = fr_FR

\chapter{Adaptation de \Tw}

\section{Mise en évidence de la Syntaxe}

Parmi beaucoup d'autres caractéristiques, \Tw{} inclut aussi la mise évidence de la syntaxe\index{syntaxe, mise en évidence}. Cela signifie que certaines choses comme les commandes \LaTeX, les environnements ou les commentaires sont colorés, soulignés ou mis en évidence d'une façon ou d'une autre. \Tw{} fournit aussi la possibilité de basculer entre différentes combinaisons de mise en évidence\footnote{ Utilisez \menu{Format}\submenu\menu{Coloration syntaxique} pour changer la combinaison de mise en évidence du document en cours, et  \menu{Édition}\submenu{Préférences\dots}\submenu\menu{Éditeur}\submenu\menu{Coloration syntaxique} pour établir le système par défaut.} et pour vos propres combinaisons. Ceci est utile si vous travaillez souvent avec des types de fichiers pour lesquels aucun système de mise en évidence n'est fourni par défaut, ou si vous voulez ajuster les combinaisons our mieux correspondre aux vôtres.

Pour modifier les combinaisons de mise en évidence, vous devez éditer le fichier texte \path{<resources>/configuration/syntax-patterns.txt}. Ce fichier contient un nombre de sections individuelles, chacune définissant une combinaison unique de mise en valeur à afficher dans la structure des menus de \Tw. Pour définir une section, écrivez uniquement le nom entouré de crochets sur une ligne. À l'évidence ces noms de doivent pas comprendre le caractère \verb|]|. Par défaut, les deux sections suivantes sont définies:
\begin{verbExample}
[LaTeX]
[ConTeXt]
\end{verbExample}
De plus, vous pouvez ajouter des commentaires dans le fichier en commençant une ligne par \verb|#|. Les lignes vides sont ignorées.

Chaque section comprend un nombre arbitraire de règles. Chacune de ces instructions consiste en trois partie: une instruction de mise en forme, un signe de correction orthographique et une expression régulière \footnote{Pour quelques détails sur les expressions régulières, voir \ref{sec:regexp}} définissant quelle partie du texte correspond. Ces parties doivent toutes être sur la même ligne et séparées par des espaces (espaces ou caractères de tabulation). Prenez par exemple la ligne qui suit à partir de la section \verb|LaTeX| par défaut:
\begin{verbExample}
red    Y    %.*
\end{verbExample}
La première partie, \verb|red|, définit le format (dans ce cas, une couleur rouge d'avant plan est spécifiée). La seconde partie, \verb|Y|, précise que la correction orthographique doit être activée pour du texte qui correspond cette règle en particulier. Parfois, il est utile de mettre \verb|N| à cet endroit pour désactiver la correction orthographique. Par exemple, si la correction était active pour les commandes de \LaTeX, la plupart des documents seraient envahis de texte souligné en rouge indiquant des mots mal orthographiés alors qu'en fait ils ne sont que des commandes spéciales. Enfin, la troisième partie spécifie que cette règle doit être appliquée à tout texte précédé de \verb|%|.

Regardons plus attentivement aux trois parties de chaque règle. Dans sa forme la plus générale, la première partie -- l'instruction de formatage -- ressemble à
\begin{verbExample}
<foreground_colour>/<background_colour>;<fontflags>
\end{verbExample}
\verb|<fontflags>| peut spécifier indépendamment des couleurs (notez cependant qu'il doit toujours être précédé d'un \verb|;|). La couleur d'arrière plan (ainsi que le \verb|/|) peut être omise, mais si vous la spécifiez, vous devez aussi spécifier la couleur d'avant plan.

Chaque couleur peut être spécifiée soit par un nom SVG\footnote{Voir \url{https://www.w3.org/TR/SVG11/types.html#ColorKeywords} pour une liste de noms valides.}, soit par une valeur hexadécimale (\verb|#rrggbb|\footnote{Parce que \verb|#| est aussi utilisé pour indiquer les commentaires si donné comme premier caractère d'une ligne, vous devez ajouter un espace, une tabulation ou quelque chose de similaire avant de spécifier une couleur hexadécimale d'avant plan.}) similaire aux documents web. Le \verb|<fontflahs>| peut être toute combinaison des lettres \verb|B| (gras), \verb|I| (italique) et \verb|U| (souligné).

Exemples d'instructions de mise en forme valides:
\begin{verbExample}
red
white/#000000
;B
blue;I
 #000000/#ffff00;U
\end{verbExample}

\section{Raccourcis clavier}
\label{sec.shortcuts}


L'utilisation des raccourcis clavier\index{raccourcis clavier} facilite grandement la saisie de la source et la gestion de celle-ci et de la vue. pour des actions fréquemment répétées, leur utilisation est beaucoup plus efficace que l'utilisation de la souris sur des boutons de barres d'outils.

Nous donnons ci-après les raccourcis clavier pour le travail dans la source et pour celui dans la vue. Notez que sous macOS, \verb|Ctrl| fait en réalité référence à la \emph{touche Commande}, qui est le modificateur habituel des raccourcis clavier. Bien que les raccourcis clavier soient spécifiés avec \verb|Ctrl|, celui-ci apparaîtra comme la \emph{touche Commande} dans les menus. (Pour faire vraiment référence à la \emph{touche Contrôle} sur la Mac, le fichier de définitions doit utiliser le nom \verb|Meta|.)

Notez que les définitions ci-dessous sont celles par défaut de la version originelle en anglais. Des langues différentes peuvent utiliser des raccourcis différents.%, ainsi que quelques changements pour la version francisée du logiciel \footnote{du moins une version de travail des raccourcis francisés, car il y a des problèmes suivant les systèmes et les habitudes des utilisateurs!}; en effet la combinaison de \verb|Ctrl| et de \verb|\| ou \verb|[| n'est pas possible sur un clavier Azerty. Lorsque les raccourcis sont les mêmes, \og --\fg{} est mis dans la colonne francisée.

Tous les raccourcis peuvent être redéfinis soit pour créer de nouveaux raccourcis ou pour modifier le raccourci existant pour se conformer aux habitudes personnelles ou changer des raccourcis non adaptés à un clavier particulier. La liste des actions\index{raccourcis clavier!actions} possibles qui peuvent être associées à un raccourci est données après la liste des raccourcis prédéfinis.

Pour définir son propre raccourci, créer un fichier \path{shortcuts.ini}\index{raccourcis clavier!shortcuts.ini} dans le sous-dossier \path{<ressources>/configuration}; s'y trouvent déjà les fichiers \path{auto-indent-patterns.txt}, \path{delimiter-pairs.txt},\dots, \path{texworks-config.txt}.

Par exemple, ce fichier pourrait contenir:
\begin{verbExample}
actionHard_Wrap = Shift+F3
actionLast_Page = Ctrl+End
actionFirst_Page = Ctrl+Home
actionWrap_Lines = F3
actionLine_Numbers = F4
actionBalance_Delimiters = F9
\end{verbExample}

La première ligne dit que la frappe de \keysequence{Maj+F3} ouvre la boîte de dialogue du retour à la ligne physique; la deuxième \keysequence{Ctrl+End} vous amène à la dernière page et \keysequence{Ctrl+Home} (troisième ligne) à la première page; avec \keysequence{F3} vous obtenez le passage à la ligne automatique ou non dans la source, avec \keysequence{F4} vous montrez/cachez les numéros de ligne et avec \keysequence{F9} vous sélectionnez le texte entre délimiteurs correspondants de la source.

\needspace{7\baselineskip}
\subsection{Raccourcis prédéfinis}\index{raccourcis clavier!prédéfinis}

Pour travailler dans la fenêtre source \footnote{Certaines combinaisons de touches ne sont pas possible sur un clavier AZERTY, aussi nous proposons du moins une version de travail des raccourcis francisés, car il y a des problèmes suivant les systèmes et les habitudes des utilisateurs!; en effet la combinaison de \verb|Ctrl| et de \verb|\| ou \verb|[| n'est pas possible. Nous proposons donc: \verb|Ctrl+]| \submenu\ \verb|Ctrl+>| (Indenter), \verb|Ctrl+[| \submenu\ \verb|Ctrl+<| (Désindenter), \verb|Ctrl+Maj+]| \submenu\ \verb|Ctrl+(| (Commenter), \verb|Ctrl+Maj+[| \submenu\ \verb|Ctrl+)| (Décommenter) et \verb|Ctrl+\| \submenu\ \verb|Ctrl+$| (Afficher/masquer le panneau de sortie)}
\input shortcutsTeXDocument.tex

\needspace{5\baselineskip}
Déplacements du curseur (maintenir \keysequence{Shift} pour sélectionner):
\begin{longtable}{Pl}
\toprule
Shortcut & Action \\
\midrule \endhead
$\rightarrow$      & 1 caractère à droite\\
Ctrl+$\rightarrow$ & 1 mot à droite\\
$\leftarrow$       & 1 caractère à gauche\\
Ctrl+$\leftarrow$  & 1 mot à gauche\\
$\uparrow$         & 1 ligne vers le haut\\
$\downarrow$       & 1 ligne vers le bas\\
PgUp               & 1 écran vers le haut\\
PgDown             & 1 écran vers le bas\\
Home               & Début de ligne\\
Ctrl+Home          & Début de document\\
End                & Fin de ligne\\
Ctrl+End           & Fin de document\\
\bottomrule
\end{longtable}

Pour travailler dans la vue .pdf:
\input shortcutsPDFDocument.tex

\subsection{Actions par ordre alphabétique}
\index{actions!ordre alphabétique}
\begin{longtable}{QQ}
\toprule
actionAbout\_Scripts                & actionPageMode\_Single \\
actionAbout\_TW                     & actionPageMode\_TwoPagesContinuous \\
actionActual\_Size                  & actionPaste \\
actionApply\_to\_Selection          & actionPlace\_on\_Left \\
actionAutoIndent\_None              & actionPlace\_on\_Right \\
actionAuto\_Follow\_Focus           & actionPreferences \\
actionBalance\_Delimiters           & actionPrevious\_Completion \\
actionClear                         & actionPrevious\_Completion\_Placeholder \\
actionClear\_Recent\_Files          & actionPrevious\_Page \\
actionClose                         & actionPrevious\_ViewRect \\
actionComment                       & actionPrintPdf \\
actionCopy                          & actionQuit\_TeXworks \\
actionCopy\_to\_Find                & actionRedo \\
actionCopy\_to\_Replace             & actionRemove\_Aux\_Files \\
actionCut                           & actionReplace \\
actionFind                          & actionReplace\_Again \\
actionFind\_Again                   & actionRevert\_to\_Saved \\
actionFind\_Selection               & actionRuler \\
actionFirst\_Page                   & actionSave \\
actionFit\_to\_Content\_Width       & actionSave\_All \\
actionFit\_to\_Width                & actionSave\_As \\
actionFit\_to\_Window               & actionScroll \\
actionFont                          & actionSelect\_All \\
actionFull\_Screen                  & actionSelect\_Text \\
actionGoToHomePage                  & actionSettings\_and\_Resources \\
actionGo\_to\_Line                  & actionShow\_Hide\_Console \\
actionGo\_to\_Page                  & actionShow\_Scripts\_Folder \\
actionGo\_to\_Preview               & actionShow\_Selection \\
actionGo\_to\_Source                & actionSide\_by\_Side \\
actionHard\_Wrap                    & actionSmartQuotes\_None \\
actionIndent                        & actionStack \\
actionInsert\_Citations             & actionSyntaxColoring\_None \\
actionJump\_To\_PDF                 & actionTile \\
actionLast\_Page                    & actionTo\_Lowercase \\
actionLine\_Numbers                 & actionTo\_Uppercase \\
actionMagnify                       & actionToggle\_Case \\
actionManage\_Scripts               & actionTypeset \\
actionNew                           & actionUncomment \\
actionNew\_from\_Template           & actionUndo \\
actionNext\_Completion              & actionUnindent \\
actionNext\_Completion\_Placeholder & actionUpdate\_Scripts \\
actionNext\_Page                    & actionWrap\_Lines \\
actionNone                          & actionWriteToMailingList \\
actionOpen                          & actionZoom\_In \\
actionPageMode\_Continuous          & actionZoom\_Out \\
\bottomrule
\end{longtable}


\subsection{Actions listées par menu}
\index{actions!by menu}

Pour la fenêtre source:
\input menuactionsTeXDocument.tex

\bigskip
Pour la fenêtre de vue:
\input menuactionsPDFDocument.tex

\subsection{Actions pour les outils de composition}
En plus des actions statiques listées ci-dessus, il y a aussi des actions pour activer un outil de composition. Toutes ces actions sont de la forme générale \verb*|actionTypesetTool<nom_outil| où \verb*|<nom_outil>| doit être remplacé de façon appropriée. Par exemple, le nom correspondant à l'outil de composition \verb*|pdfLaTeX| est \verb*|actionTypesetToolpdfLaTeX|, celui correspondant à \verb*|Mon outil| est \verb*|actionTypesetToolMon outil|.\footnote{Noter que certains caractères (comme \verb*|=| et \verb*|;| sont réservés dans les fichiers ini et doivent avoir un échappement par \verb*|%XX|, où \verb*|XX| est le code hexadécimal ASCII correspondant. Par exemple, le nom correspondant à \verb*|A=B| est \verb*|actionTypesetToolA%3DB|.}

\subsection{Actions pour les scripts}
Il y a aussi des actions pour les scripts. Elles sont dynamiques par nature, comme elles sont créées à la volée pour les scripts disponibles (qui peuvent changer quand vous installez des scripts, les retirez ou changez les paramètres.) Toutes ces actions sont de la forme \verb*|Script: <titre_script>|, où \verb*|<titre_script>| doit être remplacé de façon appropriée. Si vous avez un script qui se nomme \verb|Mon Script|, par exemple, l'action correspondante sera appelée \verb*|Script: Mon Script|.


\section{Les mots clé de complétion}
\index{complétion automatique!mots clé}\index{auto-complétion|see {complétion automatique}}
\label{sec.autocompletion}

Nous donnons ici les mots clé de complétion tels qu'ils sont fournis avec \Tw. Ils se trouvent dans les fichiers \path{tw-basic.txt}, \path{tw-context.txt} (initialement vide) et \path{tw-latex.txt} dans le répertoire \path{<ressources>\completion}.

Nous donnons les indications en trois colonnes: les deux premières montrent les mots clé, la troisième le code \AllTeX{} résultat. Dans certains cas il n'y a que le résultat, cela signifie que l'on peut commencer à entrer le code \AllTeX{} et essayer de le compléter par \keysequence{Tab}.

\newcommand{\AutoCompRet}{$\mathcal{R}$}
\newcommand{\AutoCompIns}{$\mathcal{I}$}
Lors de la complétion, le système insère des retours à la ligne et positionne le curseur de saisie au premier endroit où on doit entrer des informations pour compléter la saisie. Pour représenter les retours à la ligne nous avons utiliser \AutoCompRet{} et pour marquer le point de saisie \AutoCompIns.

\needspace{4\baselineskip}
Dès lors, une ligne telle que ``\textbackslash begin\{abstract\}{\AutoCompRet}{\AutoCompIns}{\AutoCompRet}\textbackslash end\{abstract\}•'' devrait être interprétée comme
\begin{verbExample}
\begin{abstract}

\end{abstract}•
\end{verbExample}
avec le curseur positionné dans la ligne centrale, vide.

On remarquera que les mots clé présentent une certaine cohérence, les variables mathématiques ont un mot clé commençant par \verb|x|, lorsqu'elles sont dans un environnement mathématique, quand elles sont utilisées seules dans le texte on ajoute \verb|d| devant; par exemple \verb|xa| et \verb|dxa| pour \verb|\alpha|, s'il y a une majuscule il y aura un \verb|c|, comme  \verb|xo| pour
\verb|\omega| et \verb|xco| pour \verb|\Omega|. Les mots clé d'environnements commencent par \verb|b|: \verb|bali| pour
\verb|\begin{align}| (\verb|b| mnémonique pour \verb|\begin|). Lorsque l'environnement a des options possibles, on accole un ou plusieurs \verb|o| au nom de base: \verb|bminp| pour \verb|\begin{minipage}{}..| et 
\verb|bminpo| pour \verb|\begin{minipage}[]{}..|.

%\vspace{12pt}
%
Mots clé définis dans \path{tw-basic.txt}\index{complétion automatique!tw-basic.txt} (définis dans \TeX):
%\input{tw-basic_fr.tex}
\begin{longtable}{>{\footnotesize}p{15mm}>{\footnotesize}p{15mm}>{\footnotesize}p{95mm}}
\toprule
xa   & \textbackslash xa   & \textbackslash alpha \\
xb   & \textbackslash xb   & \textbackslash beta \\
     & \textbackslash bsk  & \textbackslash bigskip \\
     &                     & \textbackslash bigskip{\AutoCompRet} \\
xch  & \textbackslash xch  & \textbackslash chi \\
xcd  & \textbackslash xcd  & \textbackslash Delta \\
xd   & \textbackslash xd   & \textbackslash delta \\
xe   & \textbackslash xe   & \textbackslash epsilon \\
xet  & \textbackslash xet  & \textbackslash eta \\
xcg  & \textbackslash xcg  & \textbackslash Gamma \\
xg   & \textbackslash xg   & \textbackslash gamma \\
     &                     & \textbackslash hskip \\
     &                     & \textbackslash indent \\
     &                     & \textbackslash input \\
xio  & \textbackslash xio  & \textbackslash iota \\
xcl  & \textbackslash xcl  & \textbackslash Lambda \\
xl   & \textbackslash xl   & \textbackslash lambda \\
     & \textbackslash msk  & \textbackslash medskip \\
     &                     & \textbackslash medskip{\AutoCompRet} \\
xm   & \textbackslash xm   & \textbackslash mu \\
     &                     & \textbackslash noindent \\
xn   & \textbackslash xn   & \textbackslash nu \\
xco  & \textbackslash xco  & \textbackslash Omega \\
xo   & \textbackslash xo   & \textbackslash omega \\
     &                     & \textbackslash par \\
xcph & \textbackslash xcph & \textbackslash Phi \\
xph  & \textbackslash xph  & \textbackslash phi \\
xcp  & \textbackslash xcp  & \textbackslash Pi \\
xp   & \textbackslash xp   & \textbackslash pi \\
xcps & \textbackslash xcps & \textbackslash Psi \\
xps  & \textbackslash xps  & \textbackslash psi \\
xr   & \textbackslash xr   & \textbackslash rho \\
     &                     & \textbackslash scriptsize \\
xcs  & \textbackslash xcs  & \textbackslash Sigma \\
xs   & \textbackslash xs   & \textbackslash sigma \\
     & \textbackslash ssk  & \textbackslash smallskip{\AutoCompRet} \\
xt   & \textbackslash xt   & \textbackslash tau \\
tex  & \textbackslash tex  & \textbackslash TeX \\
texs & \textbackslash texs & \textbackslash TeX\textbackslash \\
xcth & \textbackslash xcth & \textbackslash Theta \\
xth  & \textbackslash xth  & \textbackslash theta \\
xcu  & \textbackslash xcu  & \textbackslash Upsilon \\
xu   & \textbackslash xu   & \textbackslash upsilon \\
xve  & \textbackslash xve  & \textbackslash varepsilon \\
xvph & \textbackslash xvph & \textbackslash varphi \\
xvp  & \textbackslash xvp  & \textbackslash varpi \\
xvr  & \textbackslash xvr  & \textbackslash varrho \\
xvs  & \textbackslash xvs  & \textbackslash varsigma \\
xvth & \textbackslash xvth & \textbackslash vartheta \\
     &                     & \textbackslash vskip \\
xcx  & \textbackslash xcx  & \textbackslash Xi \\
xx   & \textbackslash xx   & \textbackslash xi \\
xz   & \textbackslash xz   & \textbackslash zeta \\
\bottomrule
\end{longtable}


\needspace{6\baselineskip}
Mots clé définis dans \path{tw-latex.txt}\index{complétion automatique!tw-latex.txt} (définis dans \LaTeX):
%\input{tw-latex_fr.tex}
\begin{longtable}{>{\footnotesize}p{15mm}>{\footnotesize}p{15mm}>{\footnotesize}p{95mm}}
\toprule
ncol            & \textbackslash ncol      &  \& \\
dd              & \textbackslash dd        & \textbackslash ( {\AutoCompIns} \textbackslash )• \\
                & \textbackslash adc       & \textbackslash addtocounter\{{\AutoCompIns}\}\{•\} \\
adcount         &                          & \textbackslash addtocounter\{{\AutoCompIns}\}\{•\}{\AutoCompRet} \\
                & \textbackslash adl       & \textbackslash addtolength\{{\AutoCompIns}\}\{•\} \\
adlen           &                          & \textbackslash addtolength\{{\AutoCompIns}\}\{•\}{\AutoCompRet} \\
                &                          & \textbackslash author\{{\AutoCompIns}\}{\AutoCompRet} \\
                &                          & \textbackslash begin\{ \\
babs            & \textbackslash babs      & \textbackslash begin\{abstract\}{\AutoCompRet}{\AutoCompIns}{\AutoCompRet}\textbackslash end\{abstract\}• \\
balis           & \textbackslash balis     & \textbackslash begin\{align*\}{\AutoCompRet}{\AutoCompIns}{\AutoCompRet}\textbackslash end\{align*\}• \\
baliats         & \textbackslash baliats   & \textbackslash begin\{alignat*\}\{{\AutoCompIns}\}{\AutoCompRet}•{\AutoCompRet}\textbackslash end\{alignat*\}• \\
baliat          & \textbackslash baliat    & \textbackslash begin\{alignat\}\{{\AutoCompIns}\}{\AutoCompRet}•{\AutoCompRet}\textbackslash end\{alignat\}• \\
baliedat        & \textbackslash baliedat  & \textbackslash begin\{alignedat\}{\AutoCompRet}{\AutoCompIns}{\AutoCompRet}\textbackslash end\{alignedat\}• \\
baliedato       & \textbackslash baliedato & \textbackslash begin\{alignedat\}[{\AutoCompIns}]{\AutoCompRet}•{\AutoCompRet}\textbackslash end\{alignedat\}• \\
balied          & \textbackslash balied    & \textbackslash begin\{aligned\}\{{\AutoCompIns}\}{\AutoCompRet}•{\AutoCompRet}\textbackslash end\{aligned\}• \\
bali            & \textbackslash bali      & \textbackslash begin\{align\}{\AutoCompRet}{\AutoCompIns}{\AutoCompRet}\textbackslash end\{align\}• \\
bapp            & \textbackslash bapp      & \textbackslash begin\{appendix\}{\AutoCompRet}{\AutoCompIns}{\AutoCompRet}\textbackslash end\{appendix\}• \\
barr            &                          & \textbackslash begin\{array\}{\AutoCompRet}{\AutoCompIns}{\AutoCompRet}\textbackslash end\{array\}• \\
bbmat           & \textbackslash bbmat     & \textbackslash begin\{bmatrix\}{\AutoCompRet}{\AutoCompIns}{\AutoCompRet}\textbackslash end\{bmatrix\}• \\
bcase           & \textbackslash bcase     & \textbackslash begin\{cases\}{\AutoCompRet}{\AutoCompIns}{\AutoCompRet}\textbackslash end\{cases\}• \\
bcent           & \textbackslash bcent     & \textbackslash begin\{center\}{\AutoCompRet}{\AutoCompIns}{\AutoCompRet}\textbackslash end\{center\}• \\
bcenum          & \textbackslash bcenum    & \textbackslash begin\{compactenum\}{\AutoCompRet}\textbackslash item{\AutoCompRet}{\AutoCompIns}{\AutoCompRet}\textbackslash end\{compactenum\}• \\
bcenumo         & \textbackslash bcenumo   & \textbackslash begin\{compactenum\}[{\AutoCompIns}]{\AutoCompRet}\textbackslash item{\AutoCompRet}•{\AutoCompRet}\textbackslash end\{compactenum\}• \\
bcitem          & \textbackslash bcitem    & \textbackslash begin\{compactitem\}{\AutoCompRet}\textbackslash item{\AutoCompRet}{\AutoCompIns}{\AutoCompRet}\textbackslash end\{compactitem\}• \\
bcitemo         & \textbackslash bcitemo   & \textbackslash begin\{compactitem\}[{\AutoCompIns}]{\AutoCompRet}\textbackslash item{\AutoCompRet}•{\AutoCompRet}\textbackslash end\{compactitem\}• \\
bdes            & \textbackslash bdes      & \textbackslash begin\{description\}{\AutoCompRet}\textbackslash item[{\AutoCompIns}]{\AutoCompRet}•{\AutoCompRet}\textbackslash end\{description\}• \\
bdoc            & \textbackslash bdoc      & \textbackslash begin\{document\}{\AutoCompRet}{\AutoCompRet}{\AutoCompIns}{\AutoCompRet}{\AutoCompRet}\textbackslash end\{document\} \\
benu            & \textbackslash benu      & \textbackslash begin\{enumerate\}{\AutoCompRet}\textbackslash item{\AutoCompRet}{\AutoCompIns}{\AutoCompRet}\textbackslash end\{enumerate\}• \\
benuo           & \textbackslash benuo     & \textbackslash begin\{enumerate\}[{\AutoCompIns}]{\AutoCompRet}\textbackslash item{\AutoCompRet}•{\AutoCompRet}\textbackslash end\{enumerate\}• \\
beqns           & \textbackslash beqns     & \textbackslash begin\{eqnarray*\}{\AutoCompRet}{\AutoCompIns}{\AutoCompRet}\textbackslash end\{eqnarray*\}• \\
beqn            & \textbackslash beqn      & \textbackslash begin\{eqnarray\}{\AutoCompRet}{\AutoCompIns}{\AutoCompRet}\textbackslash end\{eqnarray\}• \\
bequs           & \textbackslash bequs     & \textbackslash begin\{equation*\}{\AutoCompRet}{\AutoCompIns}{\AutoCompRet}\textbackslash end\{equation*\}• \\
bequ            & \textbackslash bequ      & \textbackslash begin\{equation\}{\AutoCompRet}{\AutoCompIns}{\AutoCompRet}\textbackslash end\{equation\}• \\
bfig            & \textbackslash bfig      & \textbackslash begin\{figure\}{\AutoCompRet}{\AutoCompIns}{\AutoCompRet}\textbackslash end\{figure\}• \\
bfigo           & \textbackslash bfigo     & \textbackslash begin\{figure\}[{\AutoCompIns}]{\AutoCompRet}•{\AutoCompRet}\textbackslash end\{figure\}• \\
bflaligs        & \textbackslash bflaligs  & \textbackslash begin\{flalign*\}{\AutoCompRet}{\AutoCompIns}{\AutoCompRet}\textbackslash end\{flalign*\}• \\
bflalig         & \textbackslash bflalig   & \textbackslash begin\{flalign\}{\AutoCompRet}{\AutoCompIns}{\AutoCompRet}\textbackslash end\{flalign\}• \\
bfll            & \textbackslash bfll      & \textbackslash begin\{flushleft\}{\AutoCompRet}{\AutoCompIns}{\AutoCompRet}\textbackslash end\{flushleft\}• \\
bflr            & \textbackslash bflr      & \textbackslash begin\{flushright\}{\AutoCompRet}{\AutoCompIns}{\AutoCompRet}\textbackslash end\{flushright\}• \\
bgaths          & \textbackslash bgaths    & \textbackslash begin\{gather*\}{\AutoCompRet}{\AutoCompIns}{\AutoCompRet}\textbackslash end\{gather*\}• \\
bgathed         & \textbackslash bgathed   & \textbackslash begin\{gathered\}{\AutoCompRet}{\AutoCompIns}{\AutoCompRet}\textbackslash end\{gathered\}• \\
bgathedo        & \textbackslash bgathedo  & \textbackslash begin\{gathered\}[{\AutoCompIns}]{\AutoCompRet}•{\AutoCompRet}\textbackslash end\{gathered\}• \\
bgath           & \textbackslash bgath     & \textbackslash begin\{gather\}{\AutoCompRet}{\AutoCompIns}{\AutoCompRet}\textbackslash end\{gather\}• \\
bite            & \textbackslash bite      & \textbackslash begin\{itemize\}{\AutoCompRet}\textbackslash item{\AutoCompRet}{\AutoCompIns}{\AutoCompRet}\textbackslash end\{itemize\}• \\
biteo           & \textbackslash biteo     & \textbackslash begin\{itemize\}[{\AutoCompIns}]{\AutoCompRet}\textbackslash item{\AutoCompRet}•{\AutoCompRet}\textbackslash end\{itemize\}• \\
blett           & \textbackslash blett     & \textbackslash begin\{letter\}\{{\AutoCompIns}\}{\AutoCompRet}•{\AutoCompRet}\textbackslash end\{letter\}• \\
blist           & \textbackslash blist     & \textbackslash begin\{list\}\{{\AutoCompIns}\}\{•\}{\AutoCompRet}\textbackslash item{\AutoCompRet}•{\AutoCompRet}\textbackslash end\{list\}• \\
bminpo          & \textbackslash bminpo    & \textbackslash begin\{minipage\}[{\AutoCompIns}]\{•\}{\AutoCompRet}•{\AutoCompRet}\textbackslash end\{minipage\}• \\
bminp           & \textbackslash bminp     & \textbackslash begin\{minipage\}\{{\AutoCompIns}\}{\AutoCompRet}•{\AutoCompRet}\textbackslash end\{minipage\}• \\
bmults          & \textbackslash bmults    & \textbackslash begin\{multline*\}{\AutoCompRet}{\AutoCompIns}{\AutoCompRet}\textbackslash end\{multline*\}• \\
bmult           & \textbackslash bmult     & \textbackslash begin\{multline\}{\AutoCompRet}{\AutoCompIns}{\AutoCompRet}\textbackslash end\{multline\}• \\
bpict           & \textbackslash bpict     & \textbackslash begin\{picture\}{\AutoCompRet}{\AutoCompIns}{\AutoCompRet}\textbackslash end\{picture\}• \\
bpmat           & \textbackslash bpmat     & \textbackslash begin\{pmatrix\}{\AutoCompRet}{\AutoCompIns}{\AutoCompRet}\textbackslash end\{pmatrix\}• \\
bquot           & \textbackslash bquot     & \textbackslash begin\{quotation\}{\AutoCompRet}{\AutoCompIns}{\AutoCompRet}\textbackslash end\{quotation\}• \\
bquo            & \textbackslash bquo      & \textbackslash begin\{quote\}{\AutoCompRet}{\AutoCompIns}{\AutoCompRet}\textbackslash end\{quote\}• \\
bsplit          & \textbackslash bsplit    & \textbackslash begin\{split\}{\AutoCompRet}{\AutoCompIns}{\AutoCompRet}\textbackslash end\{split\}• \\
bsubeq          & \textbackslash bsubeq    & \textbackslash begin\{subequations\}{\AutoCompRet}{\AutoCompIns}{\AutoCompRet}\textbackslash end\{subequations\}• \\
btabb           & \textbackslash btabb     & \textbackslash begin\{tabbing\}{\AutoCompRet}{\AutoCompIns}{\AutoCompRet}\textbackslash end\{tabbing\}• \\
btabls          & \textbackslash btabls    & \textbackslash begin\{table*\}{\AutoCompRet}{\AutoCompIns}{\AutoCompRet}\textbackslash end\{table*\}• \\
btbls           & \textbackslash btbls     & \textbackslash begin\{table*\}{\AutoCompRet}{\AutoCompIns}{\AutoCompRet}\textbackslash end\{table*\}• \\
btablso         & \textbackslash btablso   & \textbackslash begin\{table*\}[{\AutoCompIns}]{\AutoCompRet}•{\AutoCompRet}\textbackslash end\{table*\}• \\
btblso          & \textbackslash btblso    & \textbackslash begin\{table*\}[{\AutoCompIns}]{\AutoCompRet}•{\AutoCompRet}\textbackslash end\{table*\}• \\
btabl           & \textbackslash btabl     & \textbackslash begin\{table\}{\AutoCompRet}{\AutoCompIns}{\AutoCompRet}\textbackslash end\{table\}• \\
btbl            & \textbackslash btbl      & \textbackslash begin\{table\}{\AutoCompRet}{\AutoCompIns}{\AutoCompRet}\textbackslash end\{table\}• \\
btablo          & \textbackslash btablo    & \textbackslash begin\{table\}[{\AutoCompIns}]{\AutoCompRet}•{\AutoCompRet}\textbackslash end\{table\}• \\
btblo           & \textbackslash btblo     & \textbackslash begin\{table\}[{\AutoCompIns}]{\AutoCompRet}•{\AutoCompRet}\textbackslash end\{table\}• \\
btabs           & \textbackslash btabs     & \textbackslash begin\{tabular*\}\{{\AutoCompIns}\}\{•\}{\AutoCompRet}•{\AutoCompRet}\textbackslash end\{tabular*\}• \\
btabx           & \textbackslash btabx     & \textbackslash begin\{tabularx\}\{{\AutoCompIns}\}\{•\}{\AutoCompRet}•{\AutoCompRet}\textbackslash end\{tabularx\}• \\
btab            & \textbackslash btab      & \textbackslash begin\{tabular\}\{{\AutoCompIns}\}{\AutoCompRet}•{\AutoCompRet}\textbackslash end\{tabular\}• \\
bbib            & \textbackslash bbib      & \textbackslash begin\{thebibliography\}\{{\AutoCompIns}\}{\AutoCompRet}\textbackslash bibitem\{•\}{\AutoCompRet}•{\AutoCompRet}\textbackslash end\{thebibliography\}• \\
bindex          & \textbackslash bindex    & \textbackslash begin\{theindex\}{\AutoCompRet}{\AutoCompIns}{\AutoCompRet}\textbackslash end\{theindex\}• \\
btheo           & \textbackslash btheo     & \textbackslash begin\{theorem\}{\AutoCompRet}{\AutoCompIns}{\AutoCompRet}\textbackslash end\{theorem\}• \\
btitpg          & \textbackslash btitpg    & \textbackslash begin\{titlepage\}{\AutoCompRet}{\AutoCompIns}{\AutoCompRet}\textbackslash end\{titlepage\}• \\
btrivl          & \textbackslash btrivl    & \textbackslash begin\{trivlist\}{\AutoCompRet}{\AutoCompIns}{\AutoCompRet}\textbackslash end\{trivlist\}• \\
bvarw           & \textbackslash bvarw     & \textbackslash begin\{varwidth\}\{{\AutoCompIns}\}{\AutoCompRet}•{\AutoCompRet}\textbackslash end\{varwidth\}• \\
bverb           & \textbackslash bverb     & \textbackslash begin\{verbatim\}{\AutoCompRet}{\AutoCompIns}{\AutoCompRet}\textbackslash end\{verbatim\}• \\
bvers           & \textbackslash bvers     & \textbackslash begin\{verse\}{\AutoCompRet}{\AutoCompIns}{\AutoCompRet}\textbackslash end\{verse\}• \\
bfd             &                          & \textbackslash bfseries \\
bibitemo        &                          & \textbackslash bibitem[{\AutoCompIns}]\{•\}{\AutoCompRet}• \\
bibitem         &                          & \textbackslash bibitem\{{\AutoCompIns}\}{\AutoCompRet}• \\
bibstyle        & \textbackslash bibstyle  & \textbackslash bibliographystyle\{{\AutoCompIns}\} \\
biblio          &                          & \textbackslash bibliography\{{\AutoCompIns}\} \\
                &                          & \textbackslash boxed\{{\AutoCompIns}\} \\
                &                          & \textbackslash caption\{{\AutoCompIns}\}{\AutoCompRet} \\
                &                          & \textbackslash cdots \\
center          &                          & \textbackslash centering \\
                &                          & \textbackslash chapter\{{\AutoCompIns}\} \\
chap            &                          & \textbackslash chapter\{{\AutoCompIns}\}{\AutoCompRet} \\
                &                          & \textbackslash cite\{{\AutoCompIns}\} \\
                &                          & \textbackslash cline\{{\AutoCompIns}\} \\
                &                          & \textbackslash date\{{\AutoCompIns}\}{\AutoCompRet} \\
                &                          & \textbackslash ddddot\{{\AutoCompIns}\} \\
                &                          & \textbackslash dddot\{{\AutoCompIns}\} \\
                &                          & \textbackslash ddots \\
                &                          & \textbackslash ddot\{{\AutoCompIns}\} \\
                &                          & \textbackslash documentclass[{\AutoCompIns}]\{•\}{\AutoCompRet} \\
                &                          & \textbackslash documentclass\{{\AutoCompIns}\}{\AutoCompRet} \\
                &                          & \textbackslash dots \\
                &                          & \textbackslash dotsb \\
                &                          & \textbackslash dotsc \\
                &                          & \textbackslash dotsi \\
                &                          & \textbackslash dotsm \\
                &                          & \textbackslash dotso \\
emd             &                          & \textbackslash em \\
em              &                          & \textbackslash emph\{{\AutoCompIns}\} \\
                &                          & \textbackslash end\{{\AutoCompIns}\}{\AutoCompRet} \\
                &                          & \textbackslash eqref\{{\AutoCompIns}\} \\
                &                          & \textbackslash fboxrule\{{\AutoCompIns}\} \\
                &                          & \textbackslash fboxsep\{{\AutoCompIns}\} \\
fbox            &                          & \textbackslash fbox\{{\AutoCompIns}\} \\
                &                          & \textbackslash footnotesize \\
foot            &                          & \textbackslash footnote\{{\AutoCompIns}\} \\
frac            &                          & \textbackslash frac\{{\AutoCompIns}\}\{•\} \\
fboxoo          & \textbackslash fboxoo    & \textbackslash framebox[{\AutoCompIns}][•]\{•\} \\
fboxo           & \textbackslash fboxo     & \textbackslash framebox[{\AutoCompIns}]\{•\} \\
hw              &                          & \textbackslash headwidth \\
                &                          & \textbackslash hline{\AutoCompRet} \\
                &                          & \textbackslash hspace*\{{\AutoCompIns}\} \\
                &                          & \textbackslash hspace\{{\AutoCompIns}\} \\
incgo           &                          & \textbackslash includegraphics[{\AutoCompIns}]\{•\}{\AutoCompRet} \\
incg            &                          & \textbackslash includegraphics\{{\AutoCompIns}\}{\AutoCompRet} \\
                &                          & \textbackslash include\{{\AutoCompIns}\}{\AutoCompRet} \\
                &                          & \textbackslash input\{{\AutoCompIns}\}{\AutoCompRet} \\
                &                          & \textbackslash intertext\{{\AutoCompIns}\} \\
                &                          & \textbackslash item{\AutoCompRet}{\AutoCompIns} \\
ito             &                          & \textbackslash item[{\AutoCompIns}]{\AutoCompRet}• \\
itd             &                          & \textbackslash itshape \\
lbl             & \textbackslash lbl       & \textbackslash label\{{\AutoCompIns}\} \\
                &                          & \textbackslash large \\
                &                          & \textbackslash Large \\
latex           & \textbackslash latex     & \textbackslash LaTeX \\
latexs          & \textbackslash latexs    & \textbackslash LaTeX\textbackslash \\
latexe          & \textbackslash latexe    & \textbackslash LaTeXe \\
latexes         & \textbackslash latexes   & \textbackslash LaTeXe\textbackslash \\
                &                          & \textbackslash ldots \\
                &                          & \textbackslash listfiles{\AutoCompRet} \\
listf           & \textbackslash listf     & \textbackslash listoffigures{\AutoCompRet} \\
listt           & \textbackslash listt     & \textbackslash listoftables{\AutoCompRet} \\
                &                          & \textbackslash makeatletter \\
                &                          & \textbackslash makeatother \\
mboxoo          & \textbackslash mboxoo    & \textbackslash makebox[{\AutoCompIns}][•]\{•\} \\
mboxo           & \textbackslash mboxo     & \textbackslash makebox[{\AutoCompIns}]\{•\} \\
mpar            & \textbackslash mpar      & \textbackslash marginpar\{{\AutoCompIns}\} \\
mbf             & \textbackslash mbf       & \textbackslash mathbf\{{\AutoCompIns}\} \\
mcal            & \textbackslash mcal      & \textbackslash mathcal\{{\AutoCompIns}\} \\
mit             & \textbackslash mit       & \textbackslash mathit\{{\AutoCompIns}\} \\
mnorm           & \textbackslash mnorm     & \textbackslash mathnormal\{{\AutoCompIns}\} \\
mrm             & \textbackslash mrm       & \textbackslash mathrm\{{\AutoCompIns}\} \\
msf             & \textbackslash msf       & \textbackslash mathsf\{{\AutoCompIns}\} \\
mtt             & \textbackslash mtt       & \textbackslash mathtt\{{\AutoCompIns}\} \\
mbox            &                          & \textbackslash mbox\{{\AutoCompIns}\} \\
mdd             &                          & \textbackslash mdseries \\
multc           & \textbackslash multc     & \textbackslash multicolumn\{{\AutoCompIns}\}\{•\}\{•\} \\
multic          &                          & \textbackslash multicolumn\{{\AutoCompIns}\}\{•\}\{•\} \\
nct             &                          & \textbackslash newcolumntype\{{\AutoCompIns}\}\{•\} \\
newct           &                          & \textbackslash newcolumntype\{{\AutoCompIns}\}\{•\} \\
ncmoo           &                          & \textbackslash newcommand\{{\AutoCompIns}\}[•][•]\{•\}{\AutoCompRet} \\
newcoo          &                          & \textbackslash newcommand\{{\AutoCompIns}\}[•][•]\{•\}{\AutoCompRet} \\
ncmo            &                          & \textbackslash newcommand\{{\AutoCompIns}\}[•]\{•\}{\AutoCompRet} \\
newco           &                          & \textbackslash newcommand\{{\AutoCompIns}\}[•]\{•\}{\AutoCompRet} \\
ncm             &                          & \textbackslash newcommand\{{\AutoCompIns}\}\{•\}{\AutoCompRet} \\
newc            &                          & \textbackslash newcommand\{{\AutoCompIns}\}\{•\}{\AutoCompRet} \\
nenvoo          &                          & \textbackslash newenvironment\{{\AutoCompIns}\}[•][•]\{•\}\{•\}{\AutoCompRet} \\
neweoo          &                          & \textbackslash newenvironment\{{\AutoCompIns}\}[•][•]\{•\}\{•\}{\AutoCompRet} \\
nenvo           &                          & \textbackslash newenvironment\{{\AutoCompIns}\}[•]\{•\}\{•\}{\AutoCompRet} \\
neweo           &                          & \textbackslash newenvironment\{{\AutoCompIns}\}[•]\{•\}\{•\}{\AutoCompRet} \\
nenv            &                          & \textbackslash newenvironment\{{\AutoCompIns}\}\{•\}\{•\}{\AutoCompRet} \\
newe            &                          & \textbackslash newenvironment\{{\AutoCompIns}\}\{•\}\{•\}{\AutoCompRet} \\
newlen          &                          & \textbackslash newlength\{{\AutoCompIns}\}{\AutoCompRet} \\
nlen            &                          & \textbackslash newlength\{{\AutoCompIns}\}{\AutoCompRet} \\
newlin          &                          & \textbackslash newline{\AutoCompRet} \\
nline           &                          & \textbackslash newline{\AutoCompRet} \\
newpg           &                          & \textbackslash newpage{\AutoCompRet} \\
npg             & \textbackslash npg       & \textbackslash newpage{\AutoCompRet} \\
                &                          & \textbackslash newtheorem\{{\AutoCompIns}\}[•]\{•\}{\AutoCompRet} \\
                &                          & \textbackslash newtheorem\{{\AutoCompIns}\}\{•\}{\AutoCompRet} \\
                &                          & \textbackslash newtheorem\{{\AutoCompIns}\}\{•\}[•]{\AutoCompRet} \\
                &                          & \textbackslash nocite\{{\AutoCompIns}\} \\
                &                          & \textbackslash normalsize \\
                &                          & \textbackslash pagebreak{\AutoCompRet} \\
pgref           &                          & \textbackslash pageref\{{\AutoCompIns}\} \\
pgs             &                          & \textbackslash pagestyle\{{\AutoCompIns}\}{\AutoCompRet} \\
pars            &                          & \textbackslash paragraph*\{{\AutoCompIns}\}{\AutoCompRet} \\
paro            &                          & \textbackslash paragraph[{\AutoCompIns}]\{•\}{\AutoCompRet} \\
par             &                          & \textbackslash paragraph\{{\AutoCompIns}\}{\AutoCompRet} \\
parboxo         &                          & \textbackslash parbox[{\AutoCompIns}]\{•\}\{•\} \\
pboxo           & \textbackslash pboxo     & \textbackslash parbox[{\AutoCompIns}]\{•\}\{•\} \\
parbox          &                          & \textbackslash parbox\{{\AutoCompIns}\}\{•\} \\
pbox            & \textbackslash pbox      & \textbackslash parbox\{{\AutoCompIns}\}\{•\} \\
rboxoo          & \textbackslash rboxoo    & \textbackslash raisebox\{{\AutoCompIns}\}[•][•]\{•\} \\
rboxo           & \textbackslash rboxo     & \textbackslash raisebox\{{\AutoCompIns}\}[•]\{•\} \\
rbox            & \textbackslash rbox      & \textbackslash raisebox\{{\AutoCompIns}\}\{•\} \\
ref             &                          & \textbackslash ref\{{\AutoCompIns}\} \\
rncmoo          &                          & \textbackslash renewcommand\{{\AutoCompIns}\}[•][•]\{•\}{\AutoCompRet} \\
rnewcoo         &                          & \textbackslash renewcommand\{{\AutoCompIns}\}[•][•]\{•\}{\AutoCompRet} \\
rncmo           &                          & \textbackslash renewcommand\{{\AutoCompIns}\}[•]\{•\}{\AutoCompRet} \\
rnewco          &                          & \textbackslash renewcommand\{{\AutoCompIns}\}[•]\{•\}{\AutoCompRet} \\
rncm            &                          & \textbackslash renewcommand\{{\AutoCompIns}\}\{•\}{\AutoCompRet} \\
rnewc           &                          & \textbackslash renewcommand\{{\AutoCompIns}\}\{•\}{\AutoCompRet} \\
rmc             &                          & \textbackslash rmfamily \\
                &                          & \textbackslash rule[{\AutoCompIns}]\{•\}\{•\} \\
                &                          & \textbackslash rule\{{\AutoCompIns}\}\{•\} \\
scd             &                          & \textbackslash scshape \\
secs            &                          & \textbackslash section*\{{\AutoCompIns}\}{\AutoCompRet} \\
seco            &                          & \textbackslash section[{\AutoCompIns}]\{•\}{\AutoCompRet} \\
sec             &                          & \textbackslash section\{{\AutoCompIns}\}{\AutoCompRet} \\
                &                          & \textbackslash setlength\{{\AutoCompIns}\}\{•\} \\
hw2tw           &                          & \textbackslash setlength\{\textbackslash headwidth\}\{\textbackslash textwidth\}{\AutoCompRet} \\
sfd             &                          & \textbackslash sffamily \\
sld             &                          & \textbackslash slshape \\
sqrto           & \textbackslash sqrto     & \textbackslash sqrt[{\AutoCompIns}]\{•\} \\
sqrt            & \textbackslash sqrt      & \textbackslash sqrt\{{\AutoCompIns}\} \\
stcount         &                          & \textbackslash stepcounter\{{\AutoCompIns}\}{\AutoCompRet} \\
spars           & \textbackslash spars     & \textbackslash subparagraph*\{{\AutoCompIns}\} \\
sparo           & \textbackslash sparo     & \textbackslash subparagraph[{\AutoCompIns}]\{•\} \\
spar            & \textbackslash spar      & \textbackslash subparagraph\{{\AutoCompIns}\} \\
ssecs           & \textbackslash ssecs     & \textbackslash subsection*\{{\AutoCompIns}\}{\AutoCompRet} \\
sseco           & \textbackslash sseco     & \textbackslash subsection[{\AutoCompIns}]\{•\}{\AutoCompRet} \\
ssec            & \textbackslash ssec      & \textbackslash subsection\{{\AutoCompIns}\}{\AutoCompRet} \\
sssecs          & \textbackslash sssecs    & \textbackslash subsubsection*\{{\AutoCompIns}\}{\AutoCompRet} \\
                & \textbackslash ssseco    & \textbackslash subsubsection[{\AutoCompIns}][•]{\AutoCompRet} \\
ssseco          &                          & \textbackslash subsubsection[{\AutoCompIns}]\{•\}{\AutoCompRet} \\
sssec           & \textbackslash sssec     & \textbackslash subsubsection\{{\AutoCompIns}\}{\AutoCompRet} \\
tableofcontents &                          & \textbackslash tableofcontents{\AutoCompRet} \\
toc             & \textbackslash toc       & \textbackslash tableofcontents{\AutoCompRet} \\
tilde           & \textbackslash tilde     & \textbackslash textasciitilde \\
bf              & \textbackslash bf        & \textbackslash textbf\{{\AutoCompIns}\} \\
{-}{-}{-}       &                          & \textbackslash textemdash\textbackslash \\
{-}{-}          &                          & \textbackslash textendash\textbackslash \\
it              & \textbackslash it        & \textbackslash textit\{{\AutoCompIns}\} \\
                & \textbackslash rm        & \textbackslash textrm\{{\AutoCompIns}\} \\
sc              & \textbackslash sc        & \textbackslash textsc\{{\AutoCompIns}\} \\
sf              & \textbackslash sf        & \textbackslash textsf\{{\AutoCompIns}\} \\
sl              & \textbackslash sl        & \textbackslash textsl\{{\AutoCompIns}\} \\
tt              & \textbackslash tt        & \textbackslash texttt\{{\AutoCompIns}\} \\
up              & \textbackslash up        & \textbackslash textup\{{\AutoCompIns}\} \\
tw              & \textbackslash tw        & \textbackslash textwidth \\
                &                          & \textbackslash text\{{\AutoCompIns}\} \\
                &                          & \textbackslash thanks\{{\AutoCompIns}\}{\AutoCompRet} \\
                &                          & \textbackslash title\{{\AutoCompIns}\}{\AutoCompRet} \\
ttd             &                          & \textbackslash ttfamily \\
upd             &                          & \textbackslash upshape \\
url             &                          & \textbackslash url\{{\AutoCompIns}\} \\
usepo           &                          & \textbackslash usepackage[{\AutoCompIns}]\{•\}{\AutoCompRet} \\
usep            &                          & \textbackslash usepackage\{{\AutoCompIns}\}{\AutoCompRet} \\
                &                          & \textbackslash vdots \\
                &                          & \textbackslash vspace*\{{\AutoCompIns}\}{\AutoCompRet} \\
                &                          & \textbackslash vspace\{{\AutoCompIns}\}{\AutoCompRet} \\
                &                          & \{abstract\}{\AutoCompRet}{\AutoCompIns}{\AutoCompRet}\textbackslash end\{abstract\}• \\
                &                          & \{align*\}{\AutoCompRet}{\AutoCompIns}{\AutoCompRet}\textbackslash end\{align*\}• \\
                &                          & \{alignat*\}\{{\AutoCompIns}\}{\AutoCompRet}•{\AutoCompRet}\textbackslash end\{alignat*\}• \\
                &                          & \{alignat\}\{{\AutoCompIns}\}{\AutoCompRet}•{\AutoCompRet}\textbackslash end\{alignat\}• \\
                &                          & \{alignedat\}\{{\AutoCompIns}\}{\AutoCompRet}•{\AutoCompRet}\textbackslash end\{alignedat\}• \\
                &                          & \{aligned\}{\AutoCompRet}{\AutoCompIns}{\AutoCompRet}\textbackslash end\{aligned\}• \\
                &                          & \{aligned\}[{\AutoCompIns}]{\AutoCompRet}•{\AutoCompRet}\textbackslash end\{aligned\}• \\
                &                          & \{align\}{\AutoCompRet}{\AutoCompIns}{\AutoCompRet}\textbackslash end\{align\}• \\
                &                          & \{appendix\}{\AutoCompRet}{\AutoCompIns}{\AutoCompRet}\textbackslash end\{appendix\}• \\
                &                          & \{array\}{\AutoCompRet}{\AutoCompIns}{\AutoCompRet}\textbackslash end\{array\}• \\
                &                          & \{bmatrix\}{\AutoCompRet}{\AutoCompIns}{\AutoCompRet}\textbackslash end\{bmatrix\}• \\
                &                          & \{cases\}{\AutoCompRet}{\AutoCompIns}{\AutoCompRet}\textbackslash end\{cases\}• \\
                &                          & \{center\}{\AutoCompRet}{\AutoCompIns}{\AutoCompRet}\textbackslash end\{center\}• \\
                &                          & \{compactenum\}{\AutoCompRet}\textbackslash item{\AutoCompRet}{\AutoCompIns}{\AutoCompRet}\textbackslash end\{compactenum\}• \\
                &                          & \{compactenum\}[{\AutoCompIns}]{\AutoCompRet}\textbackslash item{\AutoCompRet}•{\AutoCompRet}\textbackslash end\{compactenum\}• \\
                &                          & \{compactitem\}{\AutoCompRet}\textbackslash item{\AutoCompRet}{\AutoCompIns}{\AutoCompRet}\textbackslash end\{compactitem\}• \\
                &                          & \{compactitem\}[{\AutoCompIns}]{\AutoCompRet}\textbackslash item{\AutoCompRet}•{\AutoCompRet}\textbackslash end\{compactitem\}• \\
                &                          & \{description\}{\AutoCompRet}\textbackslash item[{\AutoCompIns}]{\AutoCompRet}•{\AutoCompRet}\textbackslash end\{description\}• \\
                &                          & \{document\}{\AutoCompRet}{\AutoCompRet}{\AutoCompIns}{\AutoCompRet}{\AutoCompRet}\textbackslash end\{document\} \\
                &                          & \{enumerate\}{\AutoCompRet}\textbackslash item{\AutoCompRet}{\AutoCompIns}{\AutoCompRet}\textbackslash end\{enumerate\}• \\
                &                          & \{enumerate\}[{\AutoCompIns}]{\AutoCompRet}\textbackslash item{\AutoCompRet}•{\AutoCompRet}\textbackslash end\{enumerate\}• \\
                &                          & \{eqnarray*\}{\AutoCompRet}{\AutoCompIns}{\AutoCompRet}\textbackslash end\{eqnarray*\}• \\
                &                          & \{eqnarray\}{\AutoCompRet}{\AutoCompIns}{\AutoCompRet}\textbackslash end\{eqnarray\}• \\
                &                          & \{equation\}{\AutoCompRet}{\AutoCompIns}{\AutoCompRet}\textbackslash end\{equation\}• \\
                &                          & \{figure\}{\AutoCompRet}{\AutoCompIns}{\AutoCompRet}\textbackslash end\{figure\}• \\
                &                          & \{figure\}[{\AutoCompIns}]{\AutoCompRet}•{\AutoCompRet}\textbackslash end\{figure\}• \\
                &                          & \{flalign*\}{\AutoCompRet}{\AutoCompIns}{\AutoCompRet}\textbackslash end\{flalign*\}• \\
                &                          & \{flalign\}{\AutoCompRet}{\AutoCompIns}{\AutoCompRet}\textbackslash end\{flalign\}• \\
                &                          & \{flushleft\}{\AutoCompRet}{\AutoCompIns}{\AutoCompRet}\textbackslash end\{flushleft\}• \\
                &                          & \{flushright\}{\AutoCompRet}{\AutoCompIns}{\AutoCompRet}\textbackslash end\{flushright\}• \\
                &                          & \{gather*\}{\AutoCompRet}{\AutoCompIns}{\AutoCompRet}\textbackslash end\{gather*\}• \\
                &                          & \{gathered\}{\AutoCompRet}{\AutoCompIns}{\AutoCompRet}\textbackslash end\{gathered\}• \\
                &                          & \{gathered\}[{\AutoCompIns}]{\AutoCompRet}•{\AutoCompRet}\textbackslash end\{gathered\}• \\
                &                          & \{gather\}{\AutoCompRet}{\AutoCompIns}{\AutoCompRet}\textbackslash end\{gather\}• \\
                &                          & \{itemize\}{\AutoCompRet}\textbackslash item{\AutoCompRet}{\AutoCompIns}{\AutoCompRet}\textbackslash end\{itemize\}• \\
                &                          & \{itemize\}[{\AutoCompIns}]{\AutoCompRet}\textbackslash item{\AutoCompRet}•{\AutoCompRet}\textbackslash end\{itemize\}• \\
                &                          & \{letter\}\{{\AutoCompIns}\}{\AutoCompRet}•{\AutoCompRet}\textbackslash end\{letter\}• \\
                &                          & \{list\}\{{\AutoCompIns}\}\{•\}{\AutoCompRet}\textbackslash item{\AutoCompRet}•{\AutoCompRet}\textbackslash end\{list\}• \\
                &                          & \{minipage\}[{\AutoCompIns}]\{•\}{\AutoCompRet}•{\AutoCompRet}\textbackslash end\{minipage\}• \\
                &                          & \{minipage\}\{{\AutoCompIns}\}{\AutoCompRet}•{\AutoCompRet}\textbackslash end\{minipage\}• \\
                &                          & \{multline*\}{\AutoCompRet}{\AutoCompIns}{\AutoCompRet}\textbackslash end\{multline*\}• \\
                &                          & \{multline\}{\AutoCompRet}{\AutoCompIns}{\AutoCompRet}\textbackslash end\{multline\}• \\
                &                          & \{picture\}{\AutoCompRet}{\AutoCompIns}{\AutoCompRet}\textbackslash end\{picture\}• \\
                &                          & \{pmatrix\}{\AutoCompRet}{\AutoCompIns}{\AutoCompRet}\textbackslash end\{pmatrix\}• \\
                &                          & \{quotation\}{\AutoCompRet}{\AutoCompIns}{\AutoCompRet}\textbackslash end\{quotation\}• \\
                &                          & \{quote\}{\AutoCompRet}{\AutoCompIns}{\AutoCompRet}\textbackslash end\{quote\}• \\
                &                          & \{split\}{\AutoCompRet}{\AutoCompIns}{\AutoCompRet}\textbackslash end\{split\}• \\
                &                          & \{subequations\}{\AutoCompRet}{\AutoCompIns}{\AutoCompRet}\textbackslash end\{subequations\}• \\
                &                          & \{tabbing\}{\AutoCompRet}{\AutoCompIns}{\AutoCompRet}\textbackslash end\{tabbing\}• \\
                &                          & \{table*\}{\AutoCompRet}{\AutoCompIns}{\AutoCompRet}\textbackslash end\{table*\}• \\
                &                          & \{table*\}[{\AutoCompIns}]{\AutoCompRet}•{\AutoCompRet}\textbackslash end\{table*\}• \\
                &                          & \{table\}{\AutoCompRet}{\AutoCompIns}{\AutoCompRet}\textbackslash end\{table\}• \\
                &                          & \{table\}[{\AutoCompIns}]{\AutoCompRet}•{\AutoCompRet}\textbackslash end\{table\}• \\
                &                          & \{tabular*\}\{{\AutoCompIns}\}\{•\}{\AutoCompRet}•{\AutoCompRet}\textbackslash end\{tabular*\}• \\
                &                          & \{tabularx\}\{{\AutoCompIns}\}\{•\}{\AutoCompRet}•{\AutoCompRet}\textbackslash end\{tabularx\}• \\
                &                          & \{tabular\}\{{\AutoCompIns}\}{\AutoCompRet}•{\AutoCompRet}\textbackslash end\{tabular\}• \\
                &                          & \{thebibliography\}{\AutoCompRet}{\AutoCompIns}{\AutoCompRet}\textbackslash end\{thebibliography\}• \\
                &                          & \{theindex\}{\AutoCompRet}{\AutoCompIns}{\AutoCompRet}\textbackslash end\{theindex\}• \\
                &                          & \{theorem\}{\AutoCompRet}{\AutoCompIns}{\AutoCompRet}\textbackslash end\{theorem\}• \\
                &                          & \{titlepage\}{\AutoCompRet}{\AutoCompIns}{\AutoCompRet}\textbackslash end\{titlepage\}• \\
                &                          & \{trivlist\}{\AutoCompRet}{\AutoCompIns}{\AutoCompRet}\textbackslash end\{trivlist\}• \\
                &                          & \{varwidth\}\{{\AutoCompIns}\}{\AutoCompRet}•{\AutoCompRet}\textbackslash end\{varwidth\}• \\
                &                          & \{verbatim\}{\AutoCompRet}{\AutoCompIns}{\AutoCompRet}\textbackslash end\{verbatim\}• \\
                &                          & \{verse\}{\AutoCompRet}{\AutoCompIns}{\AutoCompRet}\textbackslash end\{verse\}• \\
\bottomrule
\end{longtable}


Mots clé définis dans \path{tw-latex-pkg.txt}\index{auto-completion!tw-latex-pkg.txt} (défini dans divers paquets):
\begin{longtable}{>{\footnotesize}p{15mm}>{\footnotesize}p{15mm}>{\footnotesize}p{95mm}}
\toprule
         &   & \textbackslash addbibresource\{{\AutoCompIns}\}{\AutoCompRet} \\
botr     &   & \textbackslash bottomrule{\AutoCompRet} \\
         &   & \textbackslash citep\{{\AutoCompIns}\} \\
         &   & \textbackslash citet\{{\AutoCompIns}\} \\
cmidr    &   & \textbackslash cmidrule({\AutoCompIns})\{•\} \\
cmidro   &   & \textbackslash cmidrule[{\AutoCompIns}](•)\{•\} \\
         &   & \textbackslash enquote\{{\AutoCompIns}\} \\
geometry &   & \textbackslash geometry\{•\} \\
href     &   & \textbackslash href\{{\AutoCompIns}\}\{•\} \\
         &   & \textbackslash midrule{\AutoCompRet} \\
         &   & \textbackslash printbibliography{\AutoCompRet} \\
         &   & \textbackslash printbibliography[{\AutoCompIns}]{\AutoCompRet} \\
topr     &   & \textbackslash toprule{\AutoCompRet} \\
         &   & \{compactenum\}{\AutoCompRet}\textbackslash item{\AutoCompRet}{\AutoCompIns}{\AutoCompRet}\textbackslash end\{compactenum\}• \\
         &   & \{compactenum\}[{\AutoCompIns}]{\AutoCompRet}\textbackslash item{\AutoCompRet}•{\AutoCompRet}\textbackslash end\{compactenum\}• \\
         &   & \{compactitem\}{\AutoCompRet}\textbackslash item{\AutoCompRet}{\AutoCompIns}{\AutoCompRet}\textbackslash end\{compactitem\}• \\
         &   & \{compactitem\}[{\AutoCompIns}]{\AutoCompRet}\textbackslash item{\AutoCompRet}•{\AutoCompRet}\textbackslash end\{compactitem\}• \\
         &   & \{tikzpicture\}{\AutoCompRet}{\AutoCompIns}{\AutoCompRet}\textbackslash end\{tikzpicture\}• \\
         &   & \{tikzpicture\}[{\AutoCompIns}]{\AutoCompRet}•{\AutoCompRet}\textbackslash end\{tikzpicture\}• \\
\bottomrule
\end{longtable}


Mots clé définis dans \path{tw-beamer.txt}\index{auto-completion!tw-beamer.txt} (défini by dans le paquet beamer):
\begin{longtable}{>{\footnotesize}p{15mm}>{\footnotesize}p{15mm}>{\footnotesize}p{95mm}}
\toprule
     &   & \textbackslash action<{\AutoCompIns}>\{•\} \\
     &   & \textbackslash againframe<{\AutoCompIns}>[•]\{•\}{\AutoCompRet} \\
     &   & \textbackslash againframe<{\AutoCompIns}>\{•\}{\AutoCompRet} \\
     &   & \textbackslash againframe\{{\AutoCompIns}\}{\AutoCompRet} \\
     &   & \textbackslash alert<{\AutoCompIns}>\{•\} \\
     &   & \textbackslash alert\{{\AutoCompIns}\} \\
     &   & \textbackslash alt<{\AutoCompIns}>\{•\}\{•\} \\
bfrm &   & \textbackslash begin\{frame\}{\AutoCompRet}{\AutoCompIns}{\AutoCompRet}\textbackslash end\{frame\}• \\
     &   & \textbackslash framesubtitle<{\AutoCompIns}>\{•\}{\AutoCompRet} \\
     &   & \textbackslash framesubtitle\{{\AutoCompIns}\}{\AutoCompRet} \\
     &   & \textbackslash frametitle<{\AutoCompIns}>[•]\{•\}{\AutoCompRet} \\
     &   & \textbackslash frametitle\{{\AutoCompIns}\}{\AutoCompRet} \\
     &   & \textbackslash framezoom<{\AutoCompIns}><•>(•,•)(•,•) \\
     &   & \textbackslash framezoom<{\AutoCompIns}><•>[•](•,•)(•,•) \\
     &   & \textbackslash includeonlyframes\{{\AutoCompIns}\}{\AutoCompRet} \\
     &   & \textbackslash includeonlylecture\{{\AutoCompIns}\}{\AutoCompRet} \\
     &   & \textbackslash invisible<{\AutoCompIns}>\{•\} \\
     &   & \textbackslash lecture\{{\AutoCompIns}\}\{•\}{\AutoCompRet} \\
     &   & \textbackslash note<{\AutoCompIns}>[•]\{•\}{\AutoCompRet} \\
     &   & \textbackslash note<{\AutoCompIns}>\{•\}{\AutoCompRet} \\
     &   & \textbackslash note\{{\AutoCompIns}\}{\AutoCompRet} \\
     &   & \textbackslash only<{\AutoCompIns}>\{•\} \\
     &   & \textbackslash onslide<{\AutoCompIns}>\{•\} \\
     &   & \textbackslash pause{\AutoCompRet} \\
     &   & \textbackslash pause[{\AutoCompIns}]{\AutoCompRet} \\
     &   & \textbackslash structure\{{\AutoCompIns}\}{\AutoCompRet} \\
     &   & \textbackslash temporal<{\AutoCompIns}>\{•\}\{•\}\{•\} \\
     &   & \textbackslash uncover<{\AutoCompIns}>\{•\} \\
     &   & \textbackslash usebeamercolor\{{\AutoCompIns}\}{\AutoCompRet} \\
     &   & \textbackslash visible<{\AutoCompIns}>\{•\} \\
     &   & \{actionenv\}<{\AutoCompIns}>{\AutoCompRet}{\AutoCompRet}\textbackslash end\{actionenv\}• \\
     &   & \{alertenv\}<{\AutoCompIns}>{\AutoCompRet}{\AutoCompRet}\textbackslash end\{alertenv\}• \\
     &   & \{altenv\}<{\AutoCompIns}>\{•\}\{•\}\{•\}\{•\}{\AutoCompRet}{\AutoCompRet}\textbackslash end\{altenv\}• \\
     &   & \{beamercolorbox\}[{\AutoCompIns}]\{•\}{\AutoCompRet}{\AutoCompRet}\textbackslash end\{beamercolorbox\}• \\
     &   & \{beamercolorbox\}\{{\AutoCompIns}\}{\AutoCompRet}{\AutoCompRet}\textbackslash end\{beamercolorbox\}• \\
     &   & \{block\}<{\AutoCompIns}>\{•\}{\AutoCompRet}{\AutoCompRet}\textbackslash end\{block\}• \\
     &   & \{block\}\{{\AutoCompIns}\}{\AutoCompRet}{\AutoCompRet}\textbackslash end\{block\}• \\
     &   & \{columns\}{\AutoCompRet}{\AutoCompIns}{\AutoCompRet}\textbackslash end\{columns\}• \\
     &   & \{columns\}[{\AutoCompIns}]{\AutoCompRet}•{\AutoCompRet}\textbackslash end\{columns\}• \\
     &   & \{column\}[{\AutoCompIns}]\{•\}{\AutoCompRet}{\AutoCompRet}\textbackslash end\{column\}• \\
     &   & \{column\}\{{\AutoCompIns}\}{\AutoCompRet}{\AutoCompRet}\textbackslash end\{column\}• \\
     &   & \{frame\}{\AutoCompRet}{\AutoCompIns}{\AutoCompRet}\textbackslash end\{frame\}• \\
     &   & \{frame\}[{\AutoCompIns}]\{•\}{\AutoCompRet}{\AutoCompRet}\textbackslash end\{frame\}• \\
     &   & \{frame\}\{{\AutoCompIns}\}{\AutoCompRet}{\AutoCompRet}\textbackslash end\{frame\}• \\
     &   & \{onlyenv\}<{\AutoCompIns}>{\AutoCompRet}{\AutoCompRet}\textbackslash end\{onlyenv\}• \\
     &   & \{overlayarea\}\{{\AutoCompIns}\}\{•\}{\AutoCompRet}{\AutoCompRet}\textbackslash end\{overlayarea\}• \\
     &   & \{overprint\}{\AutoCompRet}{\AutoCompIns}{\AutoCompRet}\textbackslash end\{overprint\}• \\
     &   & \{structureenv\}{\AutoCompRet}{\AutoCompIns}{\AutoCompRet}\textbackslash end\{structureenv\}• \\
\bottomrule
\end{longtable}


Mots clé définis dans \path{tw-context.txt}\index{auto-completion!tw-context.txt} (définis dans \ConTeXt):
\begin{longtable}{>{\footnotesize}p{15mm}>{\footnotesize}p{15mm}>{\footnotesize}p{95mm}}
\toprule
  &   & \textbackslash starttext \\
  &   & \textbackslash stoptext \\
\bottomrule
\end{longtable}


On y trouve aussi des codes d'environnements (ci-dessus) sans le \verb|\begin{| (qui est aussi un mot clé); cela permet si on a commencé d'entrer le code d'un environnement, de le compléter par \keysequence{Tab}.
